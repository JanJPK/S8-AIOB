% PWr general use template version 04.11.2018
%----------------------------------------------------------------------------------------
%	PACKAGES AND DOCUMENT CONFIGURATIONS
%----------------------------------------------------------------------------------------
\documentclass{article}
\usepackage{siunitx} % Provides the \SI{}{} and \si{} command for typesetting SI units
\usepackage{graphicx} % Required for the inclusion of images
\usepackage{natbib} % Required to change bibliography style to APA
\usepackage{amsmath} % Required for some math elements
\setlength\parindent{12pt} % Removes all indentation from paragraphs
%\usepackage{times} % Uncomment to use the Times New Roman font
\usepackage[export]{adjustbox}

% Polish language
\usepackage[utf8]{inputenc}
\usepackage{polski}
\usepackage[polish]{babel}

\usepackage[top=1in, bottom=1.25in, left=1.25in, right=1.25in]{geometry} % Margin
\usepackage{listings} % Code
\usepackage{indentfirst} % \par Paragraphs
\usepackage{multicol} % Multicolumn itemize
\usepackage{color} % Colored text

%\renewcommand{\labelenumi}{\alph{enumi}.} % Numbers in enumerate changed to a, b, c, ...

%\lstdefinestyle{sharpc}{language=[Sharp]C, frame=lr, rulecolor=\color{blue!80!black}}
%\lstdefinestyle{cpp}{language=C++, frame=lr}

%----------------------------------------------------------------------------------------
%	DOCUMENT INFORMATION
%----------------------------------------------------------------------------------------

\title{Analiza i ocena bezpieczeństwa systemów usługowych i IoT \\ Ocena skuteczności różnych metod łamania haseł \\ \textsc{Raport 1}}
\author{Jan \textsc{Pajdak} \\ Wojciech \textsc{Słowiński} \\ Maria \textsc{Filemonowicz}}
\date{\today}

\begin{document}
%\lstset{style=cpp}
\maketitle
\begin{center}
\begin{tabular}{l r}
Prowadzący: &  Dr hab. inż. Grzegorz \textsc{Kołaczek}
\end{tabular}
\end{center}
%\tableofcontents

%----------------------------------------------------------------------------------------
%	SECTION 0
%----------------------------------------------------------------------------------------

%\begin{multicols}{2}
%	\begin{itemize}
%		\item
%	\end{itemize}
%\end{multicols}

%\newpage
%\section{Title}}
%\subsection{Title}
%\par Text

%\begin{itemize}
%	\item\textit{Text} - Desc
%\end{itemize}

%\begin{enumerate}
%	\item Desc
%\end{enumerate}

%\begin{center}
%	\includegraphics[scale=0.6, center]{images\}
%\end{center}

%\begin{lstlisting}[basicstyle=\small]
%	programming code
%\end{lstlisting}

%----------------------------------------------------------------------------------------
%	SECTION 1
%----------------------------------------------------------------------------------------
\newpage
\section{Cel eksperymentu}
Hasła tekstowe to obecnie najpopularniejsza metoda uwierzytelniana używana do ograniczania dostępu do zasobów takich jak serwisy czy e-mail przez osoby nieupoważnione. Zabezpieczenia tego typu są łatwe w użyciu jednakże proste do złamania — w ramach eksperymentu skupimy się na łamaniu haseł przy użyciu programów implementujących algorytmy \textit{BFM} oraz \textit{Weira}

Eksperyment będzie przeprowadzony przy użyciu bazy realnych haseł, które następnie będą badane pod kątem odporności na złamanie przez poszczególne algorytmy.

%----------------------------------------------------------------------------------------
%	SECTION 2
%----------------------------------------------------------------------------------------
\section{Plan eksperymentu}
\subsection{Źródło danych}
Jako źródło danych wybrane zostały hasła dostępne na stronie \textit{Have I Been Pwned} \\ (\textit{https://haveibeenpwned.com/Passwords}). Baza zawiera 551 509 767 haseł, które pojawiły się w wyciekach informacji użytkowników, najczęściej z powodu niewystarczającego zabezpieczenia systemów. 

Hasła znajdują się w pojedynczym pliku tekstowym o wielkości 22.6 GB. Praca z tak wielkim plikiem jest problematyczna, więc zostanie on podzielony na mniejsze pliki, które mogą zostać łatwo wczytane do pamięci komputera.
\subsection{Metoda oceny}
%----------------------------------------------------------------------------------------
%	SECTION 3
%----------------------------------------------------------------------------------------
\newpage
\section{Przebieg eksperymentu}

%----------------------------------------------------------------------------------------
%	SECTION 4
%----------------------------------------------------------------------------------------
\newpage
\section{Wyniki}

%----------------------------------------------------------------------------------------
%	SECTION 5
%----------------------------------------------------------------------------------------
\newpage
\section{Analiza wyników}

%----------------------------------------------------------------------------------------
%	SECTION 6
%----------------------------------------------------------------------------------------
\newpage
\section{Podsumowanie}

%----------------------------------------------------------------------------------------
\end{document}
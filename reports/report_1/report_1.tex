% PWr general use template version 04.11.2018
%----------------------------------------------------------------------------------------
%	PACKAGES AND DOCUMENT CONFIGURATIONS
%----------------------------------------------------------------------------------------
\documentclass{article}
\usepackage{siunitx} % Provides the \SI{}{} and \si{} command for typesetting SI units
\usepackage{graphicx} % Required for the inclusion of images
%\usepackage{natbib} % Required to change bibliography style to APA
\usepackage{amsmath} % Required for some math elements
\setlength\parindent{12pt} % Removes all indentation from paragraphs
%\usepackage{times} % Uncomment to use the Times New Roman font
\usepackage[export]{adjustbox}

% Polish language
\usepackage[utf8]{inputenc}
\usepackage{polski}
\usepackage[polish]{babel}

\usepackage[top=1in, bottom=1.25in, left=1.25in, right=1.25in]{geometry} % Margin
\usepackage{listings} % Code
\usepackage{indentfirst} % \par Paragraphs
\usepackage{multicol} % Multicolumn itemize
\usepackage{color} % Colored text
\usepackage[square,sort,comma,numbers]{natbib}

%\renewcommand{\labelenumi}{\alph{enumi}.} % Numbers in enumerate changed to a, b, c, ...

%\lstdefinestyle{sharpc}{language=[Sharp]C, frame=lr, rulecolor=\color{blue!80!black}}
%\lstdefinestyle{cpp}{language=C++, frame=lr}

%----------------------------------------------------------------------------------------
%	DOCUMENT INFORMATION
%----------------------------------------------------------------------------------------

\title{Analiza i ocena bezpieczeństwa systemów usługowych i IoT \\ Ocena skuteczności różnych metod łamania haseł \\ \textsc{Raport 1}}
\author{Jan \textsc{Pajdak} \\ Wojciech \textsc{Słowiński} \\ Maria \textsc{Filemonowicz}}
\date{\today}

\begin{document}
	\bibliographystyle{plainnat}
	%\lstset{style=cpp}
	\maketitle
	\begin{center}
		\begin{tabular}{l r}
			Prowadzący: &  Dr hab. inż. Grzegorz \textsc{Kołaczek}
		\end{tabular}
	\end{center}
	
	\newpage
	\tableofcontents
	
	%----------------------------------------------------------------------------------------
	%	SECTION 0
	%----------------------------------------------------------------------------------------
	
	%\begin{multicols}{2}
	%	\begin{itemize}
	%		\item
	%	\end{itemize}
	%\end{multicols}
	
	%\newpage
	%\section{Title}}
	%\subsection{Title}
	%\par Text
	
	%\begin{itemize}
	%	\item\textit{Text} - Desc
	%\end{itemize}
	
	%\begin{enumerate}
	%	\item Desc
	%\end{enumerate}
	
	%\begin{center}
	%	\includegraphics[scale=0.6, center]{images\}
	%\end{center}
	
	%\begin{lstlisting}[basicstyle=\small]
	%	programming code
	%\end{lstlisting}
	
	%----------------------------------------------------------------------------------------
	%	SECTION 1
	%----------------------------------------------------------------------------------------
	\newpage
	\section{Cel eksperymentu}
	Hasła tekstowe to obecnie najpopularniejsza metoda uwierzytelniana używana do ograniczania dostępu do zasobów takich jak serwisy internetowe czy konta pocztowe przez osoby nieupoważnione. Zabezpieczenia tego typu są łatwe w użyciu jednakże proste do złamania — w ramach eksperymentu analizowane będzie łamanie haseł przy użyciu algorytmów \textit{BFM} oraz \textit{Weira}.
	
	Eksperyment będzie przeprowadzony przy użyciu bazy realnych haseł, które następnie będą badane pod kątem odporności na złamanie przez poszczególne algorytmy.
	
	%----------------------------------------------------------------------------------------
	%	SECTION 2
	%----------------------------------------------------------------------------------------
	\section{Plan eksperymentu}
	\subsection{Źródło danych}
	Jako źródło danych wybrana została baza danych znaleziona w roku 2017 przez firmę z branży cyberbezpieczeństwa - \textit{4iQ} \cite{breach}. Baza jest kompilacją informacji z 252 wycieków; zawiera loginy i hasła do ponad 1.4 miliarda kont. Całkowity rozmiar danych to 41.1 GB. Osoba odpowiedzialna za stworzenie bazy danych jest nieznana; dane zostały odkryte przez \textit{4iQ} w \textit{dark web} i można je obecnie pobrać przy użyciu sieci \textit{torrent}.
	
	Dane muszą zostać sformatowane przed użyciem ich w eksperymencie — są one porozdzielane na wiele plików oraz zawierają loginy i adresy poczty elektronicznej powiązane z kontami; te dodatkowe informacje są zbędne. Ze względu na ilość danych badany będzie podzbiór haseł.
	
	\subsection{Planowany przebieg eksperymentu}
	Algorytmy rozważane w niniejszym eksperymencie posiadają wysoką złożoność obliczeniową i czasową, przez co testowanie ich implementacji na dużych zbiorach haseł byłoby mocno ograniczone przez moc obliczeniową dzisiejszych komputerów i czas potrzebny na przeprowadzenie takich badań. Aby umożliwić przeprowadzenie eksperymentu na o wiele większej bazie haseł, wykorzystane zostaną metody pozwalające na wyliczenie liczby prób potrzebnych do odgadnięcia hasła w obu algorytmach, nazywane \textit{kalkulatorami} \cite{calc}. 
	
	\subsection{Technologie}
	Do formatowania bazy haseł wykorzystany został \textit{Python}.
	
	Algorytmy oceniające skuteczność łamania haseł w poszczególnych algorytmach zostały zaimplementowane przy użyciu \textit{Scala}.

	
	\subsection{Metoda oceny}
	\subsubsection{BFM}
	Kalkulator oceny skuteczności algorytmu \textit{BFM} działa następująco:
	\begin{enumerate}
		\item Na podstawie treningowego zbioru haseł określane są:
		\subitem Zbiór pojedynczych znaków alfabetu wraz z częstotliwością ich występowania jako pierwsza litera hasła 
		\subitem Zbiór wszystkich możliwych digramów ułożonych ze znaków w alfabecie wraz z częstotliwością ich występowania, tworzony w następujący sposób:
		\subsubitem Zakładając zbiór treningowy złożony ze znaków \textit{A, B, C}; Jeżeli znak \text{A} ma największe prawdopodobieństwo wystąpienia jako pierwszy, znak \textit{B} ma największe prawdopodobieństwo wystąpienia po \textit{A} a znak \textit{C} ma największe prawdopodobieństwo wystąpienia po \textit{B}, pierwszą próbą odgadnięcia hasła będzie \textit{ABC}.
		\item Na podstawie powstałych zbiorów określona zostaje kolejność zgadywania kolejnych znaków w haśle	
		%\item Dla właściwego zbioru haseł wyznaczana jest liczba wymaganych prób potrzebnych do jego odgadnięcia według następującego wzoru: $ (k-i)N^{L-i} $
		
		\item Dla właściwego zbioru haseł wyznaczana jest liczba wymaganych prób potrzebnych do jego odgadnięcia:
		\subitem \textit{N - długość zbioru znaków w alfabecie}
		\subitem \textit{L - długość zgadywanego hasła}
		\subitem \textit{M - minimalna długość hasła}
		\subitem \textit{CF - liczbę znaków sprawdzanych przed odgadnięciem właściwego pierwszego znaku hasła}
		\subitem \textit{DF(i) - liczba digramów sprawdzanych przed odgadnięciem właściwego i-tego znaku hasła}
		\subitem \textit{i - pozycja znaku w haśle}
		\subitem \textit{k - k-ta próba odgadnięcia hasła}
		\subitem \textit{X - liczba haseł krótszych niż zgadywane}
		\subitem \textit{Y - liczba haseł z niepoprawnie odgadniętą pierwszą literą}
		\subitem \textit{Z - liczba haseł z niepoprawnie zgadniętym i-tym znakiem}
		\subitem \textbf{\textit{Q - liczba prób potrzebnych do odgadnięcia zadanego hasła}}
		
		\subitem $ X = \sum_{l = L}^{l = M} N^l $		
		\subitem $ Y = CF * N^{L-1} $		
		\subitem $ Z = \sum_{i = 2}^{i = L} DF(i) * N^{L - (i + 2)} $		
		\subitem $ Q = X + Y + Z $
		
		\subitem \textit{Jeżeli pierwszy znak nie zostanie odgadnięty poprawnie to wiemy, że algorytm podejmie $ N^{L-1} $ prób, zanim spróbuje odgadnąć hasło z innym znakiem na pierwszej pozycji}
		\subitem \textit{Zgodnie z powyższym, dla k-tej próby odgadnięcia pierwszego znaku wiemy, że algorytm podejmie co najmniej $ (k-1)N^{L-1} $ prób odgadnięcia hasła}
		%\subitem \textbf{\textit{Ostateczna ilość prób odgadnięć to suma prób dla każdego znaku}}
	\end{enumerate}
	
	%----------------------------------------------------------------------------------------
	%	SECTION 3
	%----------------------------------------------------------------------------------------
	\newpage
	\section{Przebieg eksperymentu}
	\subsection{Przygotowanie danych}
	Dane znajdujące się w pobranej bazie są podzielone na 1 981 plików, w 54 folderach. Informacje w plikach są przechowywane w formacie \textit{nazwa\_użytkownika:hasło}. Rozmiar plików nie jest stały. Przed przystąpieniem do eksperymentu za pomocą skryptu napisanego w języku \textit{Python} dane zostały uporządkowane tak, by utworzyć łatwy do analizy zbiór haseł. Algorytm wykonuje następujące operacje dla każdego pliku wchodzącego w skład pobranej bazy:
	\begin{enumerate}
		\item Wczytuje pojedynczą linię z pliku
		\item Usuwa część linii zawierającą hasło
		\item Przechodzi do następnej linii jeżeli hasło zawiera znaki inne niż:
			\subitem cyfry
			\subitem litery z alfabetu łacińskiego
			\subitem znaki specjalne	
	\end{enumerate}	
	Tak przygotowana linia jest następnie przydzielana do kategorii i zapisana do odpowiedniego pliku. W eksperymencie rozróżniamy trzy kategorie haseł:
	\begin{enumerate}
		\item \textit{raw} - każde hasło które nie zostało odrzucone po analizie znaków
		\item \textit{regular8} - hasło o długości min. 8 znaków
		\item \textit{comprehensive8} - hasło o długości min. 8 znaków oraz zawierające co najmniej 1 dużą literę, 1 cyfrę i 1 znak specjalny
	\end{enumerate}
	Wymogi kategorii \textit{regular8} oraz \textit{comprehensive8} bazują na popularnych wymogach dotyczących haseł stosowanych w różnych serwisach internetowych. 
	
	Zaledwie \textit{0,5\%} haseł wchodzących w skład bazy spełnia wymogi \textit{comprehensive8}.
	
	%Wielkość plików zawierających dane testowe została ustalona na 500 000 haseł.
	
	\subsubsection{Zbiory treningowe}
	Zbiór treningowy to ważna część danych wejściowych, od których zależy efektywność algorytmów. Zbiory zostały utworzone z losowo wybranych haseł o identycznej kategorii co dane wejściowe. Dla danych wejściowych o rozmiarze $N$, ilość haseł w zestawie treningowym wynosi $ 0,1 * N $.
	
	
	\subsection{Weir (PCFG)}
	
	%----------------------------------------------------------------------------------------
	%	SECTION 4
	%----------------------------------------------------------------------------------------
	%\newpage
	\section{Wyniki}
	
	%----------------------------------------------------------------------------------------
	%	SECTION 5
	%----------------------------------------------------------------------------------------
	%\newpage
	\section{Analiza wyników}
	
	%----------------------------------------------------------------------------------------
	%	SECTION 6
	%----------------------------------------------------------------------------------------
	%\newpage
	\section{Podsumowanie}
	
	%----------------------------------------------------------------------------------------
	\newpage
	\bibliography{bibliography}
	
\end{document}
% PWr general use template version 04.11.2018
%----------------------------------------------------------------------------------------
%	PACKAGES AND DOCUMENT CONFIGURATIONS
%----------------------------------------------------------------------------------------
\documentclass{article}
\usepackage{siunitx} % Provides the \SI{}{} and \si{} command for typesetting SI units
\usepackage{graphicx} % Required for the inclusion of images
\usepackage{natbib} % Required to change bibliography style to APA
\usepackage{amsmath} % Required for some math elements
\setlength\parindent{12pt} % Removes all indentation from paragraphs
%\usepackage{times} % Uncomment to use the Times New Roman font
\usepackage[export]{adjustbox}

% Polish language
\usepackage[utf8]{inputenc}
\usepackage{polski}
\usepackage[polish]{babel}

\usepackage[top=1in, bottom=1.25in, left=1.25in, right=1.25in]{geometry} % Margin
\usepackage{listings} % Code
\usepackage{indentfirst} % \par Paragraphs
\usepackage{multicol} % Multicolumn itemize
\usepackage{color} % Colored text

%\renewcommand{\labelenumi}{\alph{enumi}.} % Numbers in enumerate changed to a, b, c, ...

%\lstdefinestyle{sharpc}{language=[Sharp]C, frame=lr, rulecolor=\color{blue!80!black}}
%\lstdefinestyle{cpp}{language=C++, frame=lr}

%----------------------------------------------------------------------------------------
%	DOCUMENT INFORMATION
%----------------------------------------------------------------------------------------

\title{Analiza i ocena bezpieczeństwa systemów usługowych i IoT \\ Ocena skuteczności różnych metod łamania haseł \\ \textsc{Raport 1}}
\author{Jan \textsc{Pajdak} \\ Wojciech \textsc{Słowiński} \\ Maria \textsc{Filemonowicz}}
\date{\today}

\begin{document}
%\lstset{style=cpp}
\maketitle
\begin{center}
\begin{tabular}{l r}
Prowadzący: &  Dr hab. inż. Grzegorz \textsc{Kołaczek}
\end{tabular}
\end{center}
%\tableofcontents

%----------------------------------------------------------------------------------------
%	SECTION 0
%----------------------------------------------------------------------------------------

%\begin{multicols}{2}
%	\begin{itemize}
%		\item
%	\end{itemize}
%\end{multicols}

%\newpage
%\section{Title}}
%\subsection{Title}
%\par Text

%\begin{itemize}
%	\item\textit{Text} - Desc
%\end{itemize}

%\begin{enumerate}
%	\item Desc
%\end{enumerate}

%\begin{center}
%	\includegraphics[scale=0.6, center]{images\}
%\end{center}

%\begin{lstlisting}[basicstyle=\small]
%	programming code
%\end{lstlisting}

%----------------------------------------------------------------------------------------
%	SECTION 1
%----------------------------------------------------------------------------------------
\newpage
\section{Cel eksperymentu}
Hasła tekstowe to obecnie najpopularniejsza metoda uwierzytelniana używana do ograniczania dostępu do zasobów takich jak serwisy czy e-mail przez osoby nieupoważnione. Zabezpieczenia tego typu są łatwe w użyciu jednakże proste do złamania — w ramach eksperymentu skupimy się na łamaniu haseł przy użyciu programów implementujących algorytmy \textit{BFM} oraz \textit{Weira}

Eksperyment będzie przeprowadzony przy użyciu bazy realnych haseł, które następnie będą badane pod kątem odporności na złamanie przez poszczególne algorytmy.

%----------------------------------------------------------------------------------------
%	SECTION 2
%----------------------------------------------------------------------------------------
\section{Plan eksperymentu}
\subsection{Źródło danych}
Jako źródło danych wybrana została baza danych znaleziona w roku 2017 przez firmę z branży cyberbezpieczeństwa - \textit{4iQ}. Baza to kompilacja informacji z 252 wycieków; zawiera loginy i hasła do ponad 1.4 miliarda kont. Całkowity rozmiar danych to 41.1 GB. Osoba odpowiedzialna za stworzenie bazy danych jest nieznana; dane zostały odkryte przez \textit{4iQ} w \textit{dark web} i można je obecnie pobrać przy użyciu sieci \textit{torrent}.

Dane muszą zostać sformatowane przed użyciem ich w eksperymencie — są one porozdzielane na wiele plików oraz zawierają loginy i adresy e-mail powiązane z kontami; te dodatkowe informacje są zbędne. Ze względu na ilość danych badany będzie podzbiór haseł.

\subsection{Technologie}
Do formatowania bazy haseł wykorzystany został \textit{Python}.

Algorytmy oceniające odporność haseł na łamanie zostały zaimplementowane przy użyciu \textit{Scala}.

\subsection{Metoda oceny}
\subsubsection{BFM}
Algorytm \textit{BFM} działa następująco:
\begin{enumerate}
	\item Na podstawie treningowego zbioru haseł określane są:
%		\subitem Tablica mieszająca w której klucze to znaki a wartości to prawdopodobieństwo wystąpienia jako pierwszy znak w haśle
		\subitem Prawdopodobieństwo wystąpienia jako pierwszy znak w haśle dla każdego znaku
		\subitem Kolejność zgadywania znaków: 
			\subsubitem Zakładając zbiór treningowy złożony ze znaków \textit{A, B, C}; Jeżeli znak \text{A} ma największe prawdopodobieństwo wystąpienia jako pierwszy, znak \textit{B} ma największe prawdopodobieństwo wystąpienia po \textit{A} a znak \textit{C} ma największe prawdopodobieństwo wystąpienia po \textit{B}, pierwszą próbą odgadnięcia hasła będzie \textit{ABC}.
	\item Dla właściwego zbioru haseł wyznaczana jest szacowana liczba wymaganych prób odgadnięcia według następującego wzoru: $ (k-i)^{L-1} $
		\subitem \textit{N - ilość możliwych znaków} 
		\subitem \textit{L - długość hasła}
		\subitem \textit{i - pozycja znaku w haśle}
		\subitem \textit{k - k-ta próba odgadnięcia hasła}
		\subitem \textit{Jeżeli pierwszy znak nie zostanie odgadnięty poprawnie to wiemy że algorytm podejmie $ N^{L-1} $ prób zanim spróbuje odgadnąć hasło z innym znakiem na pierwszej pozycji}
		\subitem \textit{Zgodnie z powyższym, dla k-tej próby odgadnięcia pierwszego znaku wiemy że algorytm podejmie conajmniej $ (k-1)N^{L-1} $ prób odgadnięcia hasła}
\end{enumerate}

%----------------------------------------------------------------------------------------
%	SECTION 3
%----------------------------------------------------------------------------------------
\newpage
\section{Przebieg eksperymentu}

%----------------------------------------------------------------------------------------
%	SECTION 4
%----------------------------------------------------------------------------------------
%\newpage
\section{Wyniki}

%----------------------------------------------------------------------------------------
%	SECTION 5
%----------------------------------------------------------------------------------------
%\newpage
\section{Analiza wyników}

%----------------------------------------------------------------------------------------
%	SECTION 6
%----------------------------------------------------------------------------------------
%\newpage
\section{Podsumowanie}

%----------------------------------------------------------------------------------------
%\begin{thebibliography}{9}
%\bibitem{breach}
%https://medium.com/4iqdelvedeep/1-4-billion-clear-text-credentials-discovered-in-a-single-database-3131d0a1ae14
%placeholder
%\end{thebibliography}

\end{document}